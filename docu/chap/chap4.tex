\chapter{总结与分析}
\label{chap:4}


作为目前最热门的研究领域之一,以 GAN 为代表的图像生成模型,通过对抗博弈的方式实现了对自然图像的高维复杂分布的建模,以求实现逼真的自然图像的生成。后续的研究通过优化网络结构、限制参数和渐进式训练等方式进一步提高了图像生成的性能,并稳定了模型的训练。

这些研究为一些更具实用性的研究奠定了基础,例如通过标签控制生成图像的类别以实现分类模型训练中的数据增强,或是实现不同域的图像之间的相互转换。然而,上述无条件生成模型生成的图像普遍存在分辨率低、质量差、语义不符合现实等问题。因此,Karras 等人提出了一系列新的图像生成模型架构和训练方法,实现了高分辨率、高质量的图像生成。这些图像生成模型通过训练,很好地建模了真实图像分布的结构特征和纹理细节,因此这些生成模型也可以被用作预训练模型,利用其学习到的先验知识,提升其他任务的深度学习模型的性能,例如将先验知识应用在图像超分辨率、上色、去噪等任务中,以预测更为合理和细节的图像结构和纹理特征,提高相关模型的性能。这些预训练生成模型同样可以被用于图像编辑等实际应用中:通过图像编码模型将待编辑的图像映射到隐向量空间中,可以通过简单、低维的向量编辑,实现风格互换、换脸等复杂、高维的图像编辑操作。

而最近提出 StyleGAN3 模型对于生成图像的内部表征进行了进一步优化,从而在亚像素尺度上实现了生成图像的平移和旋转不变性。一方面,这使得更为复杂的视频和动画生成变得更为现实,另一方面,StyleGAN3 的出现也使得实现超高维的视频空间到低维线性空间之间的映射、通过简单线性操作实现复杂的视频语义编辑成为了可能。

综上所述,StyleGAN 等图像模型在训练生成能力的同时,也学习到了图像结构特征和纹理细节等先验知识,使得这些预训练生成模型可以被应用于图像超分辨率、图像上色、图像去噪、图像编辑等各类实际应用中,并为相关任务带来显著的性能提升。不难看出,预训练图像生成模型具有极高的应用潜力和价值,但是如何将其更好的应用于特定任务中是一个难点,也会是本作业的研究重点。

最后本作业先是解释了 pSp、e4e 与 Restyle 的流程跟说明,并解释 Transformer 到 MetaFormer 与 PoolFormer 之间的关系与结构原理与状况后下,进行 PoolFormer 与 ReStyle-GAN 模型的结合改进,并详细说明,并解释客观条件的实验与主观条件的实验过程。