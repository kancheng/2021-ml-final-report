\chapter{绪论}
\label{chap:1}

\section{选题背景}

\subsection{预训练模型}

深度神经网络,如卷积神经网络(CNN),循环神经网络(RNN),图形神经网络(GNN)和注意力神经网络(Transformer)等已广泛应用于各种人工智能任务,包括计算机视觉、自然语言处理、语音信号处理、数据挖掘与推荐系统等。与以前主要依赖手工特征和统计方法的非神经网络模型不同,深度神经网络模型可以隐式地从数据中提取应用于特定任务的特征,从而摆脱复杂的特征工程的桎梏,降低任务的复杂度。然而,尽管深度神经网络取得了令人振奋的成功,其强大的特征提取能力是建立在足够的训练数据的基础上的。由于深度神经网络通常具有大量参数,因此在没有足够训练数据的情况下,网络容易产生过拟合的现象,造成很差的推理效果。考虑到这一问题,在研究者们致力于改进神经网络的架构的同时,也有大量的工作投入到应用于AI任务的高质量数据集的构建上。然而,为特定任务量身打造的数据集的构建既昂贵又耗时。因此,如何利用有限的人工标注数据,为特定任务训练有效的深层神经模型就成为了一个研究热点。而迁移学习和预训练模型的提出,为小样本学习提供了一种解决思路。 

人类可以学习用很少的样本来解决新问题,而不是用大量数据从头开始训练模型,主要原因在于人类可以利用以前学到的知识来处理新问题。受此启发,研究者将迁移学习建模为两个阶段:从一个或多个源任务捕获知识的预训练阶段,以及将捕获的知识迁移到目标任务的微调阶段。由于在预训练阶段获得了丰富的知识,微调阶段可以使模型能够很好地处理样本受限的目标任务。迁移学习为缓解数据缺乏的挑战提供了一种可行的方法,例如,通过在大规模人类注释视觉识别数据集ImageNet上对CNN进行预训练,得益于ImageNet中分布的强大视觉知识,使用少量数据微调这些预先训练的CNN就可以使其在下游任务中表现良好,因此,预训练技术被广泛应用于多种CV任务中,如图像分类、目标检测、图像分割和图像生成等。

\subsection{图像生成算法}

图像生成作为一个由来已久的研究课题,由于其广泛的应用,多年来一直是计算机视觉和计算机图形学的一个重要的研究领域。在神经网络出现之前,基于传统方法的图像生成和风格化算法有:

1.基于区域的图像生成技术。基于区域的图像生成技术结合了语义分割等技术,能够根据图像不同的语义区域的内容来适配生成。早期基于区域的算法利用区域的形状来指导笔触的布局,在图像的不同语义区域产生不同的笔画模式。之后,研究者又进一步提出了一种基于简单区域的图像处理算法来控制图像的生成,具体做法是通过用标准形状区域替换需要进行语义分割才能得到的复杂区域来创建简化的形状合成效果,并在生成过程中针对不同尺度的细节调整渲染区域的大小。用这种方法进行图像合成,生成的图像质量不高,并且效率很低,没有泛化能力,一次只能处理一张图像。

2. 基于样例的图像合成技术。基于样例的图像合成技术的思路是学习图像样本对之间的映射关系。这种技术是由Hertzmann[1]等首创的,他们提出了一种名为图像类比的算法框架。图像类比旨在以监督学习的方式学习源图像和目标图像之间的映射关系。图像类比的训练集包括成对的源图像和具有特定风格的图像,通过在训练集上训练,图像类比算法从样例训练对中学习从源域到目标域的变换,最终实现图像的风格转变。然而,成对的训练数据在实践中通常是不可求的,这就影响了该技术的普适性。此外,图像类比只能学习到较低层次图像特征,这也限制它了性能。
上述没有使用神经网络的图像生成算法虽然能够生成某些特定风格和内容的图像,但它们在灵活性、风格多样性和生成图像结构的准确性上存在很大局限,因此,需要新的算法来解决这些问题,由此基于深度学习的图像生成技术应运而生。而基于深度学习的图像生成分为基于GAN的图像生成和基于优化的图像生成两种。

(1) 基于优化的图像生成

随着卷积神经网络成为计算机视任务的主流框架,Gatys[2]等人首先尝试了使用卷积神经网络对自然图像进行生成,提出了使用在分类任务上预训练好的卷积神经网络作为特征抽取器,分别用其浅层和深层特征响应对图像的内容和风格进行统计建模,来指导图像生成。值得注意的是,这也是最早将预训练技术应用到图像生成领域的范例。他们的实验结果表明,在分类任务上预训练好的卷积神经网络能够从任意自然图像中提取内容信息,从艺术作品中提取风格信息。基于这一发现,Gatys等人首次提出利用预训练的分类卷积神经网络的输出特征来重组自然图像的内容和艺术作品的风格,通过反向传播不断优化图像,使生成图像的特征分布逐渐逼近卷积神经网络的特征响应的分布。他们提出的算法开创了利用预训练网络进行图像生成的先河,引起了学术界和工业界的广泛关注,许多科研人员进行了大量的研究来改进或扩展这种新颖的图像生成算法,由此产生了两个主要分支:

a. 基于优化图像的在线图像生成算法:它的基本思想是首先用卷积神经网络从风格和内容图像中分别提取风格信息和内容信息,再将它们重组得到优化目标,然后通过反向传播等优化算法迭代地重建图像以匹配优化目标,最后生成风格化的图像。一般来说,不同的基于图像优化的生成算法具有相同的原理,区别仅仅在于对视觉特征建模的方式。这种算法因为每次只训练一张图像,所以训练出的图像效果较好,但效率太低,并且不具备泛化能力,因为整个训练过程是在图像上进行的,不能得到通用的模型。

b. 基于优化模型的离线图像生成方法:为了解决在线优化的效率问题和泛化能力问题,Johnson等提出了在训练好 VGG 网络前加上一个卷积神经网络,固定 VGG 网络的参数,训练卷积神经网络,使其能够对输入图像进行重建。这种思路实际上是将预训练的VGG提取的特征作为损失函数来训练卷积神经网络的参数。通过这种方式训练好的卷积神经网络能够实时转换图像,具有极高的效率和较好的泛化能力。延续这种思路,后续诞生出了用一种模型生成具有多种风格的图像的算法以及用一种模型生成任意风格的图像的算法。

(2) 基于生成对抗网络的图像合成算法

基于生成对抗网络的图像合成算法属于基于神经网络的图像合成方法的一种,它有两种生成方式:随机噪声生成和条件生成。由于随机生成难以控制,难以应用到实际任务中,因此本作业主要研究条件生成,即给模型输入一幅图像,让模型合成另一幅具有该作业需要的性质的图像。基于GAN的图像合成的思路也是训练一个深度神经网络,使其能够将输入图像转换成目标风格图像。它与一般的深度图像合成方法的区别在于训练方式的不同,生成对抗网络需要同时训练一对生成器和判别器,生成器负责实现图像的合成,判别器负责对生成效果进行评价,二者通过对抗训练达到纳什均衡,最终能够得到最优的生成器,也就是本作业需要的模型。
常用的基于生成对抗网络的条件图像生成架构有 pix2pix, CycleGAN 等。pix2pix 的生成器采用 U-Net 结构,对输入图像进行下采样后再进行上采样,并且在对应层之间加入了跳跃连接;判别器采用一种称为PatchGAN的结构,用卷积神经网络对输入图片进行下采样,最后得到的特征图代表输入图像的评分矩阵,特征图上的每一点对应其感受野上的图像块的评分。采用 pix2pix 进行人图像合成,生成的图像具有很高的质量,但其缺陷在于需要成对的训练数据,而成对的图像数据往往是很难获取的。CycleGAN 的提出解决了这一问题。CycleGAN 的生成器采用ResNet结构,对图像进行下采样后用一系列残差块进行特征保持,之后再进行上采样。判别器采用了和pix2pix一样的PatchGAN结构。它和pix2pix的主要区别在于损失函数和训练方式的不同,CycleGAN中采用了循环一致性损失函数,通过训练两套生成器和判别器,用循环一致性损失函数约束两个生成器的输出,最终实现不同域间的双向迁移。采用这种方法,不需要成对的训练数据也可以实现准确高效的图像生成。

\subsection{预训练图像生成模型}

预训练图像生成模型往往指StyleGAN[6]及其改进模型,这些模型相较其他生成对抗网络而言,摒弃了相对传统的输入层,转而使用非线性映射网络,通过对风格卷积块参数进行调制的方式实现对生成图像特征的精细控制,例如人脸生成中的发型、肤色等。然而生成图像的极高质量和分辨率也意味着这类生成模型需要大量的计算资源进行训练。考虑到这类模型具有精细控制的潜力,研究人员更多地关注如何将其作为预训练模型投入到其他应用中,以提升其他任务中深度模型的性能,而非局限于无条件的生成任务。

目前而言,预训练图像生成模型的应用,一方面在于如何利用其在训练过程中学习到的知识,例如在图像超分辨率、上色、去噪等任务中引入预训练图像生成模型中的先验知识,使图像细节更丰富、使图像结构更合理;另一方面在于训练编码器将待编辑图像映射到隐向量空间中,以通过低维空间的简单操作实现图像转换等复杂的语义操作,甚至视频语义编辑等更加复杂的操作。

\section{小组成员及其分工}

\begin{enumerate}
 \item 阮洁:文献阅读、撰写报告、PPT制作、实验
 \item 周昭坤:文献阅读、撰写报告、PPT制作、实验
 \item 杜金浩:文献阅读、撰写报告、PPT制作、实验
 \item 郭博菲:文献阅读、撰写报告、PPT制作、实验
 \item 季帅健:文献阅读、撰写报告、PPT制作、实验
 \item 马子平:文献阅读、撰写报告、PPT制作、实验
 \item 潘鼎:文献阅读、撰写报告、PPT制作、实验
 \item 余旺博:文献阅读、撰写报告、PPT制作、实验
 \item 干皓丞:文献阅读、撰写报告、PPT制作、实验
 \item 徐华阳:文献阅读、撰写报告、PPT制作、实验
\end{enumerate}


\section{研究目的}

\begin{enumerate}
 \item 了解预训练模型的原理和基本应用,掌握常用的开源预训练模型的使用方法。
 \item 了解基于生成对抗网络的图像生成的基本原理,能够训练经典的生成对抗网络,如Pix2Pix, CycleGAN等,学会运用其开源的预训练模型进行图像生成。
 \item 了解预训练的分类网络在图像生成领域中的应用,例如使用预训练的VGG分类网络对图像进行内容和风格提取,实现图像风格迁移。
 \item 了解 StyleGAN 三部曲的原理及应用,能够使用其开源的预训练模型进行图像生成。
 \item 了解基于预训练 StyleGAN 的应用,如图像翻译,风格迁移,图像修复等。能够将预训练的 StyleGAN 应用到特定的底层视觉任务中。
\end{enumerate}

\section{相关文献}

\begin{description}
\item A. Hertzmann, C.E.Jacobs, N.Oliver et al. Image analogies. Proceedings of the 28th annual conference on Computer graphics and interactive techniques. ACM. 2001
\item Leon Gatys, Alexander S Ecker, Matthias Bethge. Proceedings of the IEEE conference on computer vision and pattern recognition. 2016
\item J.Johnson, A.Alahi, L.Fei-Fei. Perceptual losses for real-time style transfer and super-resolution. European Conference on Computer Vision. 2016
\item Gu J, Shen Y, Zhou B. Image processing using multi-code gan prior[C]. Proceedings of the IEEE/CVF conference on computer vision and pattern recognition. 2020: 3012-3021.
\item Alaluf Y, Patashnik O, Cohen-Or D. Restyle: A residual-based stylegan encoder via iterative refinement[C]. Proceedings of the IEEE/CVF International Conference on Computer Vision. 2021: 6711-6720.
\item Ian J.Goodfellow, Jean Pouget-Abadie, Mehdi Mirza, Bing Xu, David Warde-Farley, Sherjil Ozair, Aaron C.Courville, Yoshua Bengio: Generative Adversarial Nets. NIPS 2014: 2672-2680 
\item Alec Radford, Luke Metz, Soumith Chintala:Unsupervised Representation Learning with Deep Convolutional Generative Adversarial Networks. ICLR (Poster) 2016
\item Phillip Isola, Jun-Yan Zhu, Tinghui Zhou. Image-To-Image Translation With Conditional Adversarial Networks. Proceedings of the IEEE Conference on Computer Vision and Pattern Recognition (CVPR). 2017
\item Jun-Yan Zhu, Taesung Park, Phillip Isola. Unpaired Image-To-Image Translation Using Cycle-Consistent Adversarial Networks. Proceedings of the IEEE International Conference on Computer Vision (ICCV). 2017
\item Tero Karras, Samuli Laine, Timo Aila: A Style-Based Generator Architecture for Generative Adversarial Networks. CVPR 2019: 4401-4410
\item Tero Karras, Samuli Laine, Miika Aittala, Janne Hellsten, Jaakko Lehtinen, Timo Aila:Analyzing and Improving the Image Quality of StyleGAN. CoRR abs/1912.04958 2019
\item Tero Karras, Miika Aittala, Samuli Laine, Erik Härkönen, Janne Hellsten, Jaakko Lehtinen, Timo Aila: Alias-Free Generative Adversarial Networks. CoRR abs/2106.12423 2021
\item Weihao Xia, Yulun Zhang, Yujiu Yang, Jing-Hao Xue, Bolei Zhou, Ming-Hsuan Yang: GAN Inversion: A Survey. CoRR abs/2101.05278 2021
\item Omer Tov, Yuval Alaluf, Yotam Nitzan, Or Patashnik, Daniel Cohen-Or:
Designing an encoder for StyleGAN image manipulation. ACM Trans. Graph. 40(4): 133:1-133:14 2021
\item Elad Richardson, Yuval Alaluf, Or Patashnik, Yotam Nitzan, Yaniv Azar, Stav Shapiro, Daniel Cohen-Or: Encoding in Style: A StyleGAN Encoder for Image-to-Image Translation. CVPR 2021: 2287-2296
\item Yuval Alaluf, Or Patashnik, Daniel Cohen-Or:ReStyle: A Residual-Based StyleGAN Encoder via Iterative Refinement. CoRR abs/2104.02699 (2021)
\item Omer Kafri, Or Patashnik, Yuval Alaluf, Daniel Cohen-Or: StyleFusion: A Generative Model for Disentangling Spatial Segments. CoRR abs/2107.07437 2021
\item Radford A, Metz L, Chintala S. Unsupervised representation learning with deep convolutional generative adversarial networks[J]. arXiv preprint arXiv:1511.06434, 2015.
\item Arjovsky M, Chintala S, Bottou L. Wasserstein generative adversarial networks[C]. International conference on machine learning. PMLR, 2017: 214-223.
\item Gulrajani I, Ahmed F, Arjovsky M, et al. Improved training of wasserstein gans[J]. arXiv preprint arXiv:1704.00028, 2017.
\item Karras T, Aila T, Laine S, et al. Progressive growing of gans for improved quality, stability, and variation[J]. arXiv:1710.10196, 2017.
\item Mirza M, Osindero S. ,Conditional generative adversarial nets[J]. arXiv preprint, arXiv:1411.1784, 2014.
\item Antoniou A, Storkey A, Edwards H. Data augmentation generative adversarial networks[J]. arXiv preprint arXiv:1711.04340, 2017.
\item Gu J, Shen Y, Zhou B. Image processing using multi-code gan prior[C]. Proceedings of the IEEE/CVF conference on computer vision and pattern recognition. 2020: 3012-3021.
\end{description}

